%!TEX root = chi14grammatical.tex

A common task in qualitative data analysis is to characterize the usage of a linguistic entity by issuing queries over syntactic relations between words.
Previous interfaces for  searching over syntactic structures require programming-style queries. User interface research suggests that it is easier to recognize a pattern than to compose it from scratch; therefore, interfaces for non-experts should show previews of syntactic relations.  To determine what these previews should look like is an open question, an experiment was conducted with  400 participants.  We found that syntactic relations are recognized with 34\% higher accuracy when contextual examples are shown, than a baseline of naming the relations alone.  This suggests that user interfaces for syntactic search should display contextual examples of syntactic relations to help users choose between different relations while issuing queries.
