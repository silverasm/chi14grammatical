%!TEX root = chi14grammatical.tex

A common task in the humanities and social sciences, as well as in qualitative data analysis, is to characterize the usage of a word or concept. The ability to issue queries over syntactic relations between words is useful in these situations, as it allows users to see other words that have significant relationships with a query word. Previous interfaces for  searching over syntactic structures require programming-style queries and formal linguistic terminology, but programming experience and linguistic expertise are scarce. Interfaces for non-experts must therefore be able to show options for searching for different syntactic relations in ways that allow the user to recognize the relation that is being queried. We therefore investigated how to present syntactic relations between words in a recognizable way. An experiment with 400 participants found that syntactic relations are recognized with 34\% higher accuracy when contextual examples are shown, than a baseline of naming the relations alone.  This indicates that more user-friendly query interfaces for syntactic search should augment the shown options with contextual examples.
