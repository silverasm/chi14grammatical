%!TEX root = acl14grammatical.tex

A common task in qualitative data analysis is to characterize the usage of a linguistic entity by issuing queries over syntactic relations between words.
Previous interfaces for  searching over syntactic structures require programming-style queries. User interface research suggests that it is easier to recognize a pattern than to compose it from scratch; therefore, interfaces for non-experts should show previews of syntactic relations. What these previews should look like is an open question that we explored with a 400-participant Mechanical Turk experiment. We found that syntactic relations are recognized with 34\% higher accuracy when contextual examples are shown than a baseline of naming the relations alone.  This suggests that user interfaces should display contextual examples of syntactic relations to help users choose between different relations.