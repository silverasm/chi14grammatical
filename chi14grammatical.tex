\documentclass{sigchi}
\usepackage{caption, subcaption, graphicx, url}

\newcommand{\strong}[1] {\textbf{#1}}
\newcommand{\code}[1] {\texttt{#1}}

\begin{document}
\title{Evidence for Autosuggest for Syntactic Search}
\numberofauthors{2}
\author{
  \alignauthor 1st Author Name\\
    \affaddr{Affiliation}\\
    \affaddr{Address}\\
    \email{e-mail address}
  \alignauthor 2nd Author Name\\
    \affaddr{Affiliation}\\
    \affaddr{Address}\\
    \email{e-mail address}
}


\maketitle

\begin{abstract}
%!TEX root = chi14grammatical.tex

A controlled experiment with 400 participants found evidence supporting the use of auto-suggest of word context as a viable method of querying over syntactically parsed text collections.
\end{abstract}

\keywords{Search, Syntax, Grammatical Queries, Digital Humanities, Information Extraction}

\category{H.5.m.}{Information Interfaces and Presentation (e.g. HCI)}{Miscellaneous}

\section{Introduction}
%!TEX root = chi14grammatical.tex
The ability to search over grammatical relationships between words is useful in many non-scientific fields. For example, a social scientist trying to characterize different perspectives on immigration might ask how adjectives applying to `immigrant' have changed in the last 30 years. A scholar interested in gender might search a collection to find out whether different nouns enter into possessive relationships with `his' and `her' \cite{muralidharan2013supporting}. In other fields, grammatical queries can be used to develop patterns for recognizing entities in text, such as medical terms \cite{hirschman2005overview,maclean2013identifying}, and  products and organizations \cite{culotta2005reducing}, and for coding qualitative data such as survey results.

Most existing interfaces for syntactic search (querying over grammatical and syntactic structures) require structured query syntax. For example, the popular Stanford Parser includes Tregex, which allows for sophisticated regular expression search over syntactic tree structures \cite{levy2006tregex}.
%and Tsurgeon, which allows for manipulation of the trees extracted with Tregex
The Finite Structure Query tool for querying syntactically annotated corpora requires its queries to be stated in first order logic \cite{kepser2003finite}. In the Corpus Query Language \cite{jakubicek2010fast}, a query is a pattern of attribute-value pairs, where values can include regular expressions containing parse tree nodes and words.
Several approaches have adopted XML representations and the associated query language families of XPATH and SPARQL. For example, LPath augments XPath with additional tree operators to give it further expressiveness \cite{lai2010querying}.

However, most potential users do not have programming expertise, and are not likely to be at ease composing rigidly-structured queries. One survey found that even though linguists wished to make very technical  linguistic queries, 55\% of them did not know how to program \cite{soehn2008requirements}. In another \cite{gibbs_building_2012}, humanities scholars and social scientists are frequently skeptical of digital tools, because they are often difficult to use. This reduces the likelihood that existing structured-query tools for syntactic search will be usable by non-programmers \cite{ogden1983query}.

A related approach is the query-by-example work seen in the past in interfaces to database systems \cite{androutsopoulos1995natural}. For instance, the Linguist's Search Engine \cite{resnik2005web} uses a query-by-example strategy in which a user types in an initial sentence in English, and the system produces a graphical view of a parse tree as output, which the user can alter. 
The user can either click on the tree or modify the LISP expression to generalize the query. SPLICR also contains a graphical tree editor tool \cite{rehm2009sustainability}. 
According to Shneiderman and Plaisant \cite{shneiderman2010designing}, query-by-example has largely fallen out of favor as a user interface design approach. A downside of QBE is that the user must manipulate an example to arrive at the desired generalization.

More recently auto-suggest, a faster technique that does not require the manipulation of query by example, has become a widely-used approach in search user interfaces with strong support in terms of its usability \cite{anick2008longitudinal,ward2012autocomplete,jagadish2007making}. A list of selectable options is shown under the search bar, filtered to be relevant as the searcher types. Searchers can recognize and select the option that matches their information need, without having to generate the query themselves.

The success of auto-suggest depends upon showing users options they can recognize. However, we know of no prior work on how to display grammatical relations so that they can be easily recognized. One current presentation (not used with auto-suggest) is to name the relation and show blanks where the words that satisfy it would appear as in \emph{X is the subject of Y} \cite{muralidharan2013supporting}; we used this as the baseline presentation in our experiments because it employs the relation definitions found in the Stanford Dependency Parser's manual \cite{de2006generating}. Following the principle of recognition over recall, we hypothesized that showing contextualized usage examples would make the relations more recognizable.

Our results confirm that showing examples in the form of words or phrases significantly improves the accuracy with which grammatical relationships are recognized over the standard baseline of showing the relation name with blanks. Our findings also showed that clausal relationships, which span longer distances in sentences, benefited significantly more from example phrases than either of the other treatments.

These findings suggest that a query interface in which a user enters a word of interest and the system shows candidate grammatical relations augmented with examples from the text will be more successful than the baseline of simply naming the relation and showing gaps where the participating words appear.


%!TEX root = chi14grammatical.tex

% \section{Related Work}

% Trees are the traditional representation of syntactic parses, so trees are often the focus of query input for collections of syntactically parsed data.

% Several approaches have adopted XML representations and the associated query language families of XPATH and SPARQL. For example, LPath augments XPath with additional tree operators to give it further expressiveness \cite{lai2010querying}.


\section{Experiment 1: do Examples Help?}
\subsection{Hypothesis}
Our experiment's goal was to find out whether grammatical relationships could be made more recognizable by showing examples of their usage. We tested two types of examples: a list of matching words and a list of matching phrases containing the relationship. These alternatives correspond to the explicitly visible and implicitly-inferred portions of a grammatical relation. The words are explicitly visible in the text, but the grammatical relationship is implicitly inferred from contextual information such as the part of speech of the verb, the relative ordering, and any accompanying words.

Our hypothesis was the following:
\begin{quote}
	H1. Grammatical relations can be made more recognizable by showing examples of words or phrases that match.
\end{quote}


To test it, we presented participants with a series of identification tasks. In each task (Figure \ref{fig:task}), they were shown a list of sentences in which a particular grammatical relationship existed between two highlighted words. They were asked to identify which relationship it was from a list of four options.  Using a between-subjects design, we tested different strategies for presenting these options. Our goal was to see whether participants to whom we showed example usages identified the relationships more accurately than those to whom we did not.

\subsection{Variables}

\subsubsection{Presentation}

\begin{figure*}
	\begin{subfigure} {1.3\columnwidth}
			\centering
	\includegraphics[width=1.2\columnwidth]{fig/task}
	\caption{\label{fig:task} An example of an identification task in the \emph{phrases} condition for the relationship \code{amod(life, \_\_\_)} (where different adjectives modify the noun `life'). The correct answer is `adjective modifier' (4th option), and the remaining 3 options are distractors.}
	\end{subfigure}
	\qquad\qquad\qquad
	\begin{subfigure}{0.7\columnwidth}
		\begin{subfigure}{0.7\columnwidth}
				\centering
		\includegraphics[width=0.9\columnwidth]{fig/baseline-choices}
	    \caption {The same options as they appear in the \emph{baseline} condition \label{fig:baseline-choices}}
	    \end{subfigure}


	    \begin{subfigure}{0.7\columnwidth}
	    	\centering
	    	\includegraphics[width=0.9\columnwidth]{fig/words-choices}
	        \caption {The same options as they appear in the \emph{words} condition \label{fig:words-choices}}
	    \end{subfigure}
	\end{subfigure}
\caption{\label{fig:choices} The way the choices appeared in the three experiment conditions.}
\end{figure*}

We presented the choices in three different ways. The \strong{baseline} presentation was a short label using linguistic terminology (Figure \ref{fig:baseline-choices}), the \strong{words} presentation was the short label accompanied by a list of words that matched (Figure \ref{fig:words-choices}), and the \strong{phrases} presentation was the short label accompanied by a list of phrases in which that relationship surfaced (Figure(\ref{fig:task}).  Figure \ref{fig:choices} shows what the three conditions of this identification task looked like for the \code{amod(life, \_\_\_)} task.


\subsubsection{Relation Type}
English grammatical relationships have two dimensions of variability that our study design had to account for: different characteristics, and the fact that they involve words with two different functions.

First, grammatical relationships are not all the same, they vary in how familiar they are, the distance they span, and the variability of the wording with which they surface.  Some relationships, such as the adjective modifier, are taught in schools, whereas others are not. Some, such as adverbial relations, are distinctive because adverbs usually end in `ly'. Clausal complements and conjunctions can link words across whole sentences, whereas noun compounds only operate over adjacent words. Prepositional relationships used a fixed set of prepositions to link two word, but adverbial clauses can appear in almost any form.

Because of this variability, we had to test a number of different types grammatical relationships. We tested two main categories of relationships:
\begin{enumerate}
\item Clausal or long-distance relations:
	\begin{description}
		\item[advcl] Adverbial clause: \emph{  she \textbf{said} it while \textbf{smiling}}
		\item [xcomp] Open clausal complement:  \emph{I \textbf{learned} to \textbf{sing} }
		\item [ccomp] Clausal complement:  \emph{ I \textbf{thought} that I \textbf{knew} it}
		\item [rcmod] Relative clause modifier:  \emph{the \textbf{cat}, which we \textbf{rescued}, slept }
	\end{description}
\item Other relations:
		\begin{description}
			\item[nsubj] Subject of verb: \emph{\textbf{he} \textbf{threw} the ball}
			\item [dobj] Object of verb:  \emph{ he \textbf{threw} the \textbf{ball}}
			\item [amod] Adjective modifier \emph{\textbf{red} \textbf{ball}}
			\item [prep\_in]  Preposition (in): \emph{ the \textbf{water} in the \textbf{bucket}}
			\item [prep\_of]	Preposition (of):  \emph{ the \textbf{piece} of \textbf{cheese}}
			\item [conj\_and]  Conjunction (and)  \emph{ \textbf{mind} and \textbf{body}}
		\item[advmod] Adverbial modifier: \emph{  she \textbf{said} it \textbf{slowly}}
		\item [nn] Noun compound:  \emph{ \textbf{Mr.}  \textbf{Brown}}
		\end{description}
\end{enumerate}
\subsubsection{Words}
The second dimension of variability is that a relation links two words that have different functions. In the verb-subject relationship ``\emph{\textbf{he} \textbf{threw}}'', ``he'' is a noun and ``threw'' is a verb. When presenting a participant with a list of sentences containing the relationship, we therefore have several options: we could keep the relationship the same and vary the two words that are linked,  we could keep the relationship and one word the same, and vary the second, or we could keep all three the same.

We decided on the middle approach -- to fix the relationship as well as one of the words, but to test each relationship 4 times, with different words in the two different roles. For example, the verb-subject relation \code{nsubj} was tested in the following four forms:
\begin{enumerate}
	\item \code{nsubj(Ahab, \_\_\_)}:  the sentences each contained `Ahab', highlighted in yellow, as the subject of different verbs highlighted in pink.
	\item \code{nsubj(captain, \_\_\_)}

	\item \code{nsubj(\_\_\_, said)}: the sentences all contained the verb `said', highlighted in yellow, but with different subjects, highlighted in pink.
	\item \code{nsubj(\_\_\_, stood)}
\end{enumerate}

\subsubsection{Task Variables}

The tasks were all generated using the Stanford Parser on the text of \emph{Moby Dick} by Herman Melville. When parse errors appeared, we corrected them by hand.

To maximize coverage, yet keep the number of tasks reasonable (around 7 or 8 minutes), we divided the relations above into 4 task sets of 3 relations each. Each relation was tested with 4 different words, making a total of 12 tasks per participant.

The tasks were presented in the same order, and the choices were also presented in the same order: the only variation between participants was the way in which those choices were displayed. In each task, there was a `query' word and a relationship. The participants were shown list of 8 sentences containing that relationship between the query word and other words. The query word was highlighted in yellow and the matching word in pink (Figure \ref{fig:task}). Their task was to identify the relationship from list of 4 choices.

To make sure that the participants could not simply guess the right answer by pattern-matching, we ensured that there was no overlap between the list of sentences shown, and the examples shown in the choices as words or phrases.


\subsection{Participants}
There were 400 participants in total, split randomly across the 4 task sets and the 3 presentations. The ability to issue grammatical search queries is relevant to many fields outside linguistics and language study. We therefore wanted to avoid having any specific backgrounds overrepresented. To achieve this, we chose Amazon's Mechanical Turk crowdsourcing platform as a source of study participants.

Participants were paid 50 cents for completing the task, with an additional 50-cent bonus if they correctly identified 10 or more of the 12 relationships. They were informed of the possibility of the bonus before starting the task.

\subsubsection{Screening}

As is difficult to ensure the quality of effort from participants from Mechanical Turk, we included a multiple-choice screening question, `What is the third word of this sentence?"  Those that answered incorrectly were eliminated.


\subsection{Results}
\begin{figure}
\includegraphics[width=1.0\columnwidth]{fig/overall-success-rates}
\caption{
	\label{fig:overall-success-rates} Average recognition success rates for the three different presentations.
}
\end{figure}
Our results (Figure \ref{fig:overall-success-rates}) confirm H1: examples improve the recognizability of grammatical relations. Participants in the \strong{baseline} condition were significantly worse at identifying the relations than participants in conditions that showed examples (\strong{phrases} and \strong{words}).  The average success rate (where success means that the participant correctly identified the relation) in the \strong{baseline} condition was $41\%$, which is significantly\footnote{Using the Wilcoxson signed-rank test, an alternative to the standard T-test that does not assume samples are normally distributed.} less accurate than in the two example-showing conditions: \strong{words}: $52\%, (p = 0.00019)$, and \strong{phrases} condition : $55\%, (p = 0.00013)$.

The difference between the two types of examples,  \strong{phrases} and \strong{words},  was not significant overall, but the data revealed an interesting fact when they were compared across the different types  of relations (Figure \ref{fig:results-by-relation-type}). In all cases, the baseline performs worse that an example-showing presentation. However, the three different categories of relations behaved very differently with respect to whether phrases or words was better.

\begin{figure}
\includegraphics[width=1.0\columnwidth]{fig/by-relation-type}
\caption{
	\label{fig:results-by-relation-type} Average recognition success rates for the three categories of relations, by presentation.
}
\end{figure}


For the clausal relations, which operate over longer distances in sentences, the data confirmed what one might intuitively expect. Phrases, which show the usage context, significantly improved recognizability compared to the list of words or the baseline labels. The average success rate is 48\% for \strong{phrases}, which is significantly more than \strong{words}: 38\%, (p = 0.017), or \strong{baseline}: 24\%, ($p= 1.9 \times 10^9$).

For the other relations, there was no real difference between \strong{phrases} and \strong{words}, although they were both still significantly better than the baseline (words: $p=0.0063$, phrases: $p=0.023$).


\section{Discussion}
%!TEX root = chi14grammatical.tex
Our results imply that auto-suggest interfaces for syntactic search should show candidate relationships augmented with a list of phrases in which they occur. A list of phrases is the most recognizable presentation for clausal relationships, and is as good as a list of words for the other types of relations. A mockup of such a search box is shown in Figure \ref{fig:phrases-mockup}.
\begin{figure}
\centering
\includegraphics[width=0.5\columnwidth]{fig/phrases-mockup}
\caption{
	\label{fig:phrases-mockup} Mockup of auto-suggest for syntactic search on the word `life', showing the most common grammatical relations with example phrases for each.
}
\end{figure}



\bibliographystyle{acm-sigchi}
\bibliography{papers}

\end{document}


%\section{Experiment 2}
%We hypothesized that words were more helpful when the relations involved distinctive or closed-class words: adverbs are distinctive because they usually end in `ly' -- thoughtfully, helpfully, quickly, etc. Closed-class words include determiners (a, the, that, etc.), pronouns and prepositions.  Our follow-up hypothesis was:
%
%\begin{quote}
%	H2. If distinctive or closed-class words enter into a grammatical relation, then showing a list of matching words makes it easier to recognize the relation. In other cases, a list of example phrases is more helpful.
%\end{quote}
%
%We conducted a follow-up study to verify this hypothesis. This study was had the same design as the first study: a series of identification tasks in which there was a query word and a relationship. The participants  were shown a list of sentences containing that relationship between the query word and other words, their task was to identify which relationship it was, from a list of 4 choices. Except this time, instead of presenting all the choices in the same way (as a list of matching words, or a list of matching phrases) we presented each choice in the way we thought would be most useful according to our hypothesis. If the hypothesis was true, participants in this `optimal presentation' condition would outperform participants who had all the options presented in the same way.
%
%The results from this follow-up experiment  confirmed our hypothesis.
